%!TEX root = ../proteoform_nomenclature.tex
%---------------------------------------------------------------------
%	Best Practices
%---------------------------------------------------------------------

\section{Best Practices}

The best practices emphasize human readability and clarity of modification identities for publishing sequences.

\begin{enumerate}

\item In the pipe-separated list, the most descriptive element should go first to improve human readability.

\item If the identity of a modification is known, it should be listed. This improves the clarity over listing only masses or accessions.

\item Rule 6 should be used when there is only one element in the tag -- otherwise, human readability is compromised.

\item Spacing before and after each descriptor is arbitrary, and should be appropriately added to improve readability.

\item In the case of multiple key value pairs, UniMod interim names without key are recommended to come first for human readability.

\item Some Unimod interim names contain colons. When these occur it is best to include the label Unimod.

\end{enumerate}

\indent Not best practice: \texttt{SEQUK[Label:13C(3)][Acetyl]ENCE} \\
\indent Best practice: \texttt{SEQUK[Unimod:Label:13C(3)][Acetyl]ENCE} \\
