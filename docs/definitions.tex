%!TEX root = ../proteoform_nomenclature.tex
%---------------------------------------------------------------------
%	Definitions
%---------------------------------------------------------------------

\section{Definitions}

\begin{sortedlist}

\sortitem{Descriptor:\\
Member of the tag. Could be a key-value pair, or a keyless entry.}

\sortitem{Human Readable:\\
A strong emphasis is placed on human readability for proteoform names. Proteoforms should be written in a manner allowing general audience members to know exactly the sequence of amino acids and the positions of any modifications, described in as accurate detail as possible.}

\sortitem{Key:\\
An optional element of a descriptor that specifies the descriptor type. It must be followed by a colon and a value.}

\sortitem{Machine Readable:\\
Adherence to the conventions described above should facility the creation and utilization of generic parsers so that proteoforms could be exchanged between users using a computer interface.}

\sortitem{Modification:\\
Includes the addition and subtraction of specific atoms, atom combinations and/or masses, relative to the genetically encoded amino acid at a specific residue in a proteoform}

\sortitem{Proteoform:\\
A specific set of amino acids arranged in a particular order, which may be further modified (cotranslationally, posttranslationally or chemically) at designated locations.}

\sortitem{Tag:\\
The specified way of writing a localized modification. Everything between ‘[‘ and ‘]’ (inclusive). A collection of descriptors.}

\sortitem{Value:\\
Contents of a descriptor, such as the mass, chemical composition, or modification name.}

\end{sortedlist}